\section*{Abstract}

Open source software is one of the fundamental building blocks of today's
technology dependent society and is relied upon by parties ranging from large technology corporations
to individual hobbyist developers. The open question left for technologists is
how to make open source software projects more sustainable.

The rise of decentralized networks of self-organizing, self-coordinating users
incentivized by valuable cryptographic tokens enabled by Ethereum smart
contracts creates the possibility of a system with embedded economics for open
source software development that aligns the incentives of all parties. We
present two contributions that can serve as building blocks for a potentially better
solution to open source software sustainability: a command line tool that
enables a decentralized Git workflow without the need for a centralized service
like Github and a proof-of-concept blockchain based protocol for incentivizing
open source software development using a cryptographic token. Both contributions
are implemented using Ethereum smart contracts.
