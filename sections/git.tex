\section{Decentralized Git Workflow}
\label{sec:git}

\subsection{Mango}

Althought Git is a distributed version control system, it is commonly used in a
centralized manner. Developers often work using a local Git repository and
coordinate with other developers by pushing their changes to a remote Git
repository hosted somewhere. While developers can choose to use their own
servers to host remote Git repositories, it is far more common to rely on a web
service such as Github to handle hosting. Outsourcing hosting work to Github
relieves developers of additional work, but also forces developers to trust and
rely on Github.

Mango is a remote protocol for Git that uses Ethereum smart contracts for repository
access control and stores Git objects in decentralized content addressable storage
networks\cite{mango}. Moving repository control away from a centralized third party to a smart
contract also introduces the possibility of new repository features such as programatic
payments and voting mechansims. A smart contract controlled repository makes the
economic incentives for development and governance in the OpenCollab protocol possible.

\subsection{Extensions to Mango}

In order for Mango to be viable for a decentralized Git workflow, it needs
to support pull requests and issues. As a part of our contributions, we
extended the \textproc{MangoRepo} smart contract implementation by adding an
issue tracking system and pull request support.

Users create and edit issues by uploading issue contents to Swarm and mapping
the Swarm hash to an issue id in the contract.

Users can create a pull request by forking the project and initializing
a new Mango repository for the fork. The user can then make relevant changes in
the fork, push the fork using Mango and reference the \textproc{MangoRepo}
contract address for the fork in a new pull request. Maintainers can then review
a pull request by cloning the Mango repository using its contract address.

% Local Variables:
% org-ref-default-bibliography: ../bib/git.bib
% End:
