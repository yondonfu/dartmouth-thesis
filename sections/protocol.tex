\section{OpenCollab Protocol}
\label{sec:opencollab}

\subsection{Protocol Roles}

\begin{itemize}
  \item \textbf{Voters}: vote for project issues by staking tokens. The weight
    of a vote is proportional to the number of tokens staked.
  \item \textbf{Contributors}: open pull requests to resolve issues by staking
    tokens. If a contributor's pull request is merged into the project, the
    contributor earns a portion of an issue's token reward.
  \item \textbf{Maintainers}: review and merge pull requests for issues by
    staking tokens. If a maintainer successfully merges in a pull request, the
    maintainer earns a portion of the issue's token reward.
\end{itemize}

\subsection{OpenCollab Token}

The OpenCollab token (OCT) powers the OpenCollab protocol. The value offered by
the token is influence over an open source software project. Voters acquire tokens
to signal the importance of various project issues. Contributors acquire tokens to
open pull requests and earn additional tokens by making quality contributions to
a project. Maintainers acquire tokens to merge pull requests and earn additional
tokens for merging quality contributions into a project. Furthermore, the token is
used for a number of purposes in the protocol:

\begin{itemize}
  \item Used in a staking mechanism for issue voting. Voters stake tokens when
    voting for an issue. The amount of tokens staked signals the importance a
    voter places on an issue.
  \item Used in a staking mechanism for opening pull requests. Contributors
    stake a certain number of tokens when opening a pull request. If a
    contributor's pull request is closed without being merged in to the project,
    the contributor's staked tokens are destroyed. The possibility of losing
    staked tokens discourages contributors from opening pull requests unless
    they are confident about the quality of their contributions.
  \item Used in a staking mechanism for merging pull requests. Maintainers stake
    a certain number of tokens when they initiate a merge. Before a merge is
    finalized, a token holder can challenge a maintainer's merge to start a
    voting round. If token holders decide to veto a maintainer's merge, the
    maintainer's staked tokens are destroyed. The possibility of losing staked
    tokens discrouages maintainers from merging pull requests that do not
    benefit a project. The challenge and voting process for a merge is described
    in more detail in Section \ref{sec:merge}.
\end{itemize}

\subsection{Voting on Issues}

\subsection{Opening Pull Requests}

\subsection{Merging Pull Requests}
\label{sec:merge}
