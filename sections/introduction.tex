\section{Introduction}

In today's software dependent society, open source software is everywhere.
Parties ranging from large technology corporations like Google to individual
hobbyist developers use open source software as the building blocks of their own
projects. Tools that a developer once had to build from scratch are now widely
available for anyone to use for free on websites like Github. Not only can
anyone easily use these software packages, but anyone can also freely access,
inspect and alter the source code, tailoring it for his or her own specialized needs.

The implications of democratized access to quality software is wide ranging. The
open source web framework Ruby on Rails not only powers popular applications
such as Twitter and Github that millions of people rely on everyday, but it also
made web application development accessible to a broader audience by abstracting
away the details of composing together components such as HTTP request handling,
database querying and templating. Given the importance of open source software,
the task at hand for technologists is to figure out how to make open source
software projects sustainable such that organizations and individuals in the
future can continue to rely on them in the future.

The sustainability of an open source software project is tied to project health
and support. Project health is determined by how actively and adequately project
developers communicate with users such that the project
addresses the needs of the community. Project support is determined by the
availability of financial and technical resources to develop a project\cite{successOSS}. A sustainable
open source software project needs to be both healthy and supported.

A healthy and supported project optimizes the use of developer time and
attention, the scarcest resources in open source software projects.
Communication between developers and users in a healthy project informs
developers of community needs and issues to potentially focus their time and
attention on. Availability of financial and technical resources in a supported
project ensures developers are free to allocate their all of their time and attention on
project issues. Consequently, in a healthy and supported project, developers can
properly allocate their time and attention according to community needs.

If project developers fail to properly allocate their time and attention, users
might leave a project in search of alternatives that better suit
their needs. Canonical, the company behind the Linux operating system Ubuntu,
created fragmentation in the Linux community when it shipped a new version of
Ubuntu with the Unity interface rather than the standard GNOME
interface\cite{ubuntuUnity}. Canonical's failure to properly poll for user
opinion and allocate developer time and attention accordingly ultimately hurt
the Ubuntu project.

\subsection{Bounty Systems}

A proposed solution to open source software sustainability is a bounty system
for project issues such as the one operated by the website
Bountysource\cite{bountysource}. In these systems, users can attach monetary
bounties to project issues that are rewarded to contributors that successfully
resolve issues. While these bounty systems may help projects attract more
contributors, they do not take into account the incentives of maintainers. The
work done by maintainers to review and merge in code is just
as crucial as the work done by contributors. Consequently, bounty systems only
partially help with project support.

Furthermore, bounty systems do not necessarily help with
project health. Although in some cases, multiple users placing bounties on an
issue might signal the importance the community places on that particular issue,
it is also possible for malicious actors to place large bounties on issues that
would negatively impact project quality. Such a possibility can place a burden
on developers of filtering signal from noise and also introduces the possibility
of collusion - a contributor might share a large bounty if a maintainer agrees
to merge it into the codebase even if the contributed code is of poor quality.
As a result, bounty systems can actualy hamper communication
between developers and users leading to unmet community needs. Adding any form of financial compensation to open source
software projects needs to align the incentives of all parties
involved or else perverse incentives might arise leading to malicious behavior
that harms the quality of the project.

Lastly, bounty systems operated by websites like Bountysource rely on a
centralized entity to facilitate transactions. This reliance on a centralized
entity not only results in a central point of failure, but can actually be more
costly for users. For example, although users can freely transact within the bounty system, Bountysource
charges a 10\% withdrawal fee if a user wants to cash out. As a result, users
choose between giving up a portion of their monetary rewards and giving up the numerous
opportunies to use their monetary rewards for their own benefit outside the bounty system. This withdrawal free
discourages users from leaving the system which benefits Bountysource, but harms users.

\subsection{Cryptocurrencies and Blockchain Systems}

The advent of cryptocurrencies and blockchains introduce a new decentralized
paradigm for social systems. Cryptocurrencies are digital assets that rely on
cryptography to secure transactions. In general, when we use the term
cryptocurrencies, we describe decentralized
cryptocurrencies managed by a distributed network of computers as opposed to
fiat currencies that are managed by a central bank. Blockchains are the
underlying technology that make these cryptocurrencies possible. These data
structures establish the state of a system, whether it be a currency system or
otherwise, without placing trust in a single entity. We will discuss blockchains
in more detail in Section \ref{sec:blockchains}.

Blockchains can not only enable the creation of a bounty system that is not
managed by a single entity, but also allows for the creation of a system with
more complex rules that potentially offer economic support in a way that aligns the
incentives of all parties involved. With blockchains, developers can embed a set
of rules for updating system state directly into software. Instead of trusting a
centralized entity to enforce the rules, users know that the software will
enforce the rules since it is programmed to determinstically execute and respond
to a predetermined set of instructions. In a centralized paradigm, systems rely
on central entities for coordination and organization. In a decentralized
paradigm, systems are formed by a distributed network of self-coordinating and
self-organizing users that follow a common software protocol powered by a
blockchain. Protocols can economically incentivize certain actions by rewarding
users with protocol native cryptocurrencies if conditions established in the
protocol rule set are fulfilled.

Communities around open source software projects currently rely on centralized
services to coordinate and collaborate. Adding monetary rewards to a project in
a way that aligns interests of all parties is not only difficult, but also adds
middlemen and transaction costs to the system. Cryptocurrencies and
blockchains can be the building blocks for a potentially better system.

\subsection{Contributions}

The primary contributions of this thesis are the following:

\begin{itemize}
  \item A command line tool that enables a decentralized Git workflow for
    developing open source software without relying on a centralized service
    like Github (Section \ref{sec:git}).
  \item A proof-of-concept blockchain based protocol to incentivize open source
    software development (Section \ref{sec:opencollab}).
\end{itemize}

The code for the command line tool is available open source at \url{https://github.com/yondonfu/opencollab}.

The code for the set of smart contracts implementing the OpenCollab protocol is
available open source at \url{https://github.com/yondonfu/opencollab-contracts}.

The motivation behind these contributions is to push the discussion on how to
improve open source software sustainability. In particular, these contributions are
attempts to answer the following questions relating to open source software sustainability.

\begin{itemize}
  \item How can developers poll for user opinion on issue
    priorization for a project?
  \item How can a project attract regular contributors?
  \item How can maintainers be incentivized to carefully review and merge pull
    requests such that the quality of a project is upheld?
 \end{itemize}

A detailed description of these contributions can be found in Sections
\ref{sec:git} and \ref{sec:opencollab}.

% Local Variables:
% org-ref-default-bibliography: ../bib/introduction.bib
