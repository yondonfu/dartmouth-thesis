\section{Introduction}

In today's software dependent society, open source software is everywhere.
Parties ranging from large technology corporations like Google to individual
hobbyist developers use open source software as the building blocks of their own
projects. Tools that a developer once had to build from scratch are now widely
available for anyone to use for free on websites like Github. Not only can
anyone easily use these software packages, but anyone can also freely access,
inspect and alter the source code, tailoring it for his or her own specialized needs.

The implications of democratized access to quality software is wide ranging. The
open source web framework Ruby on Rails not only powers popular applications
such as Twitter and Github that millions of people rely on everyday, but it also
made web application development accessible to a broader audience by abstracting
away the details of composing together components such as HTTP request handling,
database querying and templating.

The sustainability of an open source software project is tied to project health
and support. Project health is determined by how actively and adequately project
maintainers and contributors communicate with users such that the project
addresses the needs of the community. Project support is determined by the
availability of resources to develop a project\cite{successOSS}.

Developer time and attention are the scarcest resources in open source software
projects. Users might leave a project in search of alternatives that better suit
their needs if project developers fail to properly allocate their time and
attention. Canonical, the company behind the Linux operating system Ubuntu,
created fragmentation in the Linux community when it shipped a new version of
Ubuntu with the Unity interface rather than the standard GNOME
interface\cite{ubuntuUnity}. Canonical's failure to properly poll for user
opinion and allocate developer time and attention accordingly ultimately hurt
the Ubuntu project.

\subsection{Bounty Systems}

A proposed solution to open source software sustainability is a bounty system
for issues such as the one operated by the website
Bountysource\cite{bountysource}. In these systems, users can attach monetary
bounties to project issues that are rewarded to contributors that successfuly
resolve issues.

% Local Variables:
% org-ref-default-bibliography: ../bib/introduction.bib
