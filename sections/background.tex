\section{Background}

Despite only recently receiving more mainstream attention among developers and business
people, cryptocurrencies and blockchains have a long history. The technical
foundations of cryptocurrencies and blockchains evolved from past work by
computer science researchers. Two of the most important areas of research are in
electronic cash systems and digital time-stamping.

\subsection{Electronic Cash Systems}

The advent of the Internet and the mainstream adoption of computing devices
allowed people to freely transfer data between each other. A logical extension
of online data transfer is online cash transfer. However, creating an electronic
cash system comes with a number challenges. Similar to distributed systems, a useable
electronic system also requires the ACID properties\cite{camp1995}:

\begin{itemize}
  \item Atomicity: a transaction either occurs completely or does not occur at all
  \item Consistency: relevant parties for a transaction agree on the critical
    facts of exchange
  \item Isolation: transactions do not interfere with each other
  \item Durability: if one or more computers crash, the system should be able to
    recover to the last consistent state
\end{itemize}

David Chaum attempted to design a system with some of these properties with his
DigiCash project. In DigiCash, a bank issues and verifies electronic tokens that network
users can exchange with one another\cite{camp1995}. One of the flaws of
DigitCash is that it can only offer durable transactions in exchange for a loss
anonymity. Network users must present tokens to the bank for verification or be
at risk of double spending attacks since electronic messages can be duplicated.
Additionally, this reliance on the bank creates a bottleneck for system
throughput and a central point of failure. In the scenario where an attacker
gains access to the bank's private key, the attacker can create counterfeit
tokens that are indistinguishable from valid tokens at will.

\subsection{Digital Time-Stamping}

The Internet and mainstream adoption of computing devices allowed for the
widespread digitization of all types of documents. While the digitization of
documents provided many benefits, it also came with the problem of how to
certify the existence and time of creation or change of a digital document.

In 1991, Haber and Stornetta presented a time-stamping method for digital
documents that consisted of certificates cryptographically signed by a time-stamping service. The
certificates contain the hash of the document as well as linking information from a
previous certificate which includes a hash of the previous certificate's linking
information\cite{haber1991}. The result is a hash linked chain of certificates
that prevents the faking of time-stamps.

Bayer, Haber and Stornetta extended this time-stamping method using merkle
trees. In the original time-stamping method, verification of a document
timestamp can require at most N steps by following the chain links to a
time-stamp certificate that is trustworthy\cite{bayer1993}. Instead of linking N
hashes of documents, the hash values can be stored in a merkle tree.
Participants can record the hashes of their own documents and the sibling hash
values along the path from the document hash to the root of the merkle tree.
Consequently, verification can be done in at most $\lg N$ steps by presenting the
document hash and the $\lg N$ hashes on the path to the root. This modified
time-stamping approach reduces storage requirements and verification time.

\subsection{Blockchains}

Blockchains combine learnings from previous electronic cash systems like Chaum's
DigiCash and digital time-stamping methods. A blockchain is distributed ledger
that is not controlled or managed by a central entity. The ledger is powered by
a network of connected computers that use a consensus mechanism to reach agreement over shared data\cite{whatisblockchain}.

\subsection{Cryptographic Primitives and Data Structures}

\subsection{Bitcoin}

\subsection{Ethereum}

% Local Variables:
% org-ref-default-bibliography: ../bib/background.bib
% End:
