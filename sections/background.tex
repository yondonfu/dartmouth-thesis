\section{Background}
\label{sec:background}

Cryptographic tokens and blockchains (more specifically Ethereum) serve as the foundation of the contributions
for this thesis. A more detailed discussion of blockchains can be found in
Section \ref{sec:related}.

\subsection{Cryptographic Tokens}

Although Bitcoin is often referred to as a cryptocurrency, it is also commonly
classified as a cryptographic token. We make this differentiation in terminology
to highlight the difference in usage that is associated with each term. Both
terms share the common definition of a digital asset secured by cryptography that be transferred without
the permission of its original issuer. Cryptocurrencies are primarily associated
with being mediums of exchange and stores of value. While cryptographic tokens
can also serve these roles, they also offer utility in that they provide
the holder access to a useful service. Bitcoin was the first cryptographic token
and it allowed holders to access and write to a global and immutable decentralized
database by creating transactions. Cryptographic tokens can be compared to paid
API keys that are redeemable for a service and that can be freely traded without
the permission of the original API key issuer\cite{balajiTokens}. In the
remaining sections of this thesis, we will use the term tokens to refer to
cryptographic tokens with the properties highlighted above.

Token systems rely on underlying blockchains to provide a decentralized database
that contains records of token ownership. Ownership of a token is defined by
ownership of the private key referenced by a record with the associated public key in the blockchain. The absence of a centralized
intermediary enforcing the ownership of tokens allows token holders to freely
transfer their tokens to other parties. The ability to freely trade tokens and
the ability to use tokens to access a service gives tokens a floating price on
the open market\cite{balajiTokens}. Thus, when integrated into a protocol, tokens can serve as an economic mechanism that aligns the incentives of
network users. Individuals purchase tokens on the open market to access a useful service offered by
the protocol. Purchasers become token holders that stand to benefit if the token rises in
value. These early adopters increase interest in the protocol by actively using
the service provided or building additional applications and services on top of it. Increased
protocol interest leads to increased demand for tokens so that users can access
the service offered by the protocol. The result is token value appreciation
and a new wave of token holders that are financially incentivized to increase
the value of the protocol\cite{fatprotocols}.

A key observation of these token powered protocols is that token holders and
protocol users are the same group of individuals. Consequently, the incentives of the users are
aligned such that they all want to see the value of their token holdings increase.

\subsection{Ethereum}

Token powered protocols can be built in a number of ways. As mentioned
previously, Bitcoin is an example of a token that powers the Bitcoin protocol
which offers users peer to peer value transfer as a service. The key
component of a token system is the underlying blockchain that it relies on.
Individuals can choose to create an entirely new blockchain to support their
token systems. Alternatively, they can build their token systems on existing
blockchains such as the Bitcoin blockchain. Although token systems such as
Counterparty have been built on top of the Bitcoin blockchain, the
Bitcoin blockchain and its limited built-in scripting language does not provide developers with a lot flexibility and
expressiveness when designing the mechanics of their token systems\cite{counterparty}.

Ethereum is a blockchain designed for general computation and offers a built-in
Turing complete programming language that is more flexible and expressive than the
Bitcoin scripting language allowing users to write so called \textit{smart
  contracts}\cite{ethereum}. These smart contracts can define the rules and state associated
with a token powered protocol. Users of token powered protocols interact with the smart contract and
trust the automatic enforced execution of the smart contract code which is
secured by the underlying Ethereum blockchain. Interest in token powered protocols have given rise to a community developed
standard for tokens built on top of Ethereum called ERC20\cite{erc20}.

Although security focused members of the computer science community have
expressed a fair amount of concern about the viability of Ethereum as a
blockchain used for smart contracts due to the large attack surface presented by
its Turing complete programming language, the stark reality is that the Ethereum
developer ecosystem is the most active of any other blockchain ecosystem.
Furthermore, various members of the community are actively researching methods
to better secure the Ethereum network including formal verification,
proof-of-stake as an alternative consensus algorithm to proof-of-work and smart
contract programming languages with stronger security guarantees.
Consequently, with these points in mind we decided to build the OpenCollab protocol on Ethereum.

% Local Variables:
% org-ref-default-bibliography: ../bib/background.bib
% End:
