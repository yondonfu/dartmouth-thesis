\section{Background}

Although blockchains have only recently captured the attention of academics and
business people, the technical foundations of blockchains actually have a longer
history. Blockchains use ideas rom past work by computer science
researchers in the areas of distributed consensus, electronic money
and digital time-stamping.

\subsection{Distributed Consensus}

The reliability of a distributed system depends on system proccesses to
reach consensus on particular values. We can analyze a distributed system as a
replicated state machine consisting of a state machine replicated across
multiple processes using a deterministic state transition function to map a set
of inputs and the current state to a new state\cite{faulttolerant}. State
transitions are atomic such that they either occur completely or do not occur at
all, consistent such that they must be valid mappings of inputs and the current
state to a new state and durable such that once they occur, state is permanently
updated. Thus, we can consider state transition function inputs as transactions\cite{transactions}.

A system using multiple proccesses is vulnerable to proccess faults that cause
unexpected process behavior. Faults can be broadly categorized as fail-stop
faults, where processes crash and other proccesses can detect the failure, and
Byzantine faults, where processes can exhibit arbitrary and potentially
malicious behavior\cite{faulttolerant}. The problem of handling Byzantine faults
is more complex than that of handling fail-stop faults because Byzantine faulty
processes can transmit conflicting information to other processes that might
not be immediately detected\cite{byzantine}.

Consequently, distributed systems require specific consensus algorithms to
handle Byzantine faults. One example of a Byzantine fault tolerant consensus
algorithm is Practical Byzantine Fault Tolerance (PBFT) which offers system
reliability as long as out of $n$ processes there are $\frac{n - 1}{3}$ or less
faulty processes at any given point in time\cite{pbft}. However, PBFT cannot
guarantee system reliability if $\frac{n - 1}{3}$ or more processes collude.

\subsection{Electronic Money}

The advent of the Internet and the mainstream adoption of computing devices
encouraged the development of many forms of electronic money by recording
balances electronically on devices.

In 1983, David Chaum introduced a cryptographic protocol for anonymous payments using
blind signatures. In Chaum's protocol, a bank uses its private key to sign
blinded tokens and payees accept signed tokens by clearing a
signed token with the bank\cite{blindsignatures}. The authenticity of tokens can
be guaranteed by verifying the bank's signature on tokens with the bank's public
key. The bank also does not know the identity of a payer when clearing a signed
token sent by a payee because the token was blinded to obfuscate the amount and sender the bank
originally signed the token. Chaum applied this protocol in his DigiCash project.

One of the flaws of DigiCash is that it can only offer durable
transactions in exchange for decreased anonymity. Users must present tokens to
the bank for verification or else they are vulnerable to double spending attacks
since electronic messages can easily be duplicated\cite{camp1995}. Furthermore,
reliance on the bank creates a bottleneck for system throughput and a central
point of failure. If a bank's private key is compromised, an attacker can use
the bank's private key to create counterfeit tokens.

Easily duplicated electronic messages leave electronic money systems vulnerable
to denial of service attacks. As a solution, Adam Back proposed using hashcash, a
easily verifiable, but difficult to compute cost function to mint
tokens\cite{hashcash}. The cost function or \textit{proof-of-work} is based on finding partial hash
collisions on the $k$-bit string $0^{k}$, for which the fastest known algorithm
is brute force meaning users must perform a certain amount of work in terms of
computing cycles to mint tokens. A proof-of-work requirement discourages
electronic message duplication by making message creation costly.

Nick Szabo highlighted the utility of proof-of-work for electronic money in his bit gold protocol. The protocol
uses a proof of work function to compute a string of bits that is timestamped in a distributed property title registry\cite{bitgold}. Users can
verify owernship of a string of bits in the title registry. In contrast with
DigiCash, bit gold allows valuable bits to be created,
transferred and stored without depending on a trusted third party. However, a
system implementing such a protocol was never implemented in practice.

\subsection{Digital Time-Stamping}

Widespread digitization of all types of documents brought many benefits to
society, but also introduced the question of how to prove the existence and time of creation or change of a
digital document.

In 1991, Haber and Stornetta presented a time-stamping method for digital
documents that consisted of certificates cryptographically signed by a time-stamping service. The
certificates contain the hash of the document as well as linking information from a
previous certificate which includes a hash of the previous certificate's linking
information\cite{haber1991}. The result is a hash linked chain of certificates
that prevents the faking of time-stamps.

Bayer, Haber and Stornetta extended this time-stamping method using merkle
trees. In the original time-stamping method, verification of a document
timestamp can require at most $N$ steps by following the chain links to a
time-stamp certificate that is trustworthy\cite{bayer1993}. Instead of linking $N$
hashes of documents, the hash values can be stored in a merkle tree.
Participants can record the hashes of their own documents and the sibling hash
values along the path from the document hash to the root of the merkle tree.
Consequently, verification can be done in at most $\lg N$ steps by presenting the
document hash and the $\lg N$ hashes on the path to the root. This modified
time-stamping approach reduces storage requirements and verification time.

\subsection{Blockchains}
\label{sec:blockchains}

Blockchains combine learnings from past work in distributed consensus,
electronic money and digital time-stamping. A blockchain is a type of distributed system with two key defining
characteristics. The first characteristic is that transactions are grouped into blocks. A
common optimization reminiscient of Bayer, Haber and
Stronetta's digital time-stamping method using merkle trees is to store the root of a merkle tree that contains
transactions in the block header. This optimization allows clients to easily
verify transactions solely using downloaded block headers without storing all
the actual transactions. The second characteristic is that blocks
are linked by cryptographic hashes. As demonstrated by Haber and Stornetta's
work with hash linked digital timestamps, a hash linked chain of blocks prevents
tampering of blocks unless an adversary has majority control of the system such
that it can rewrite the entire hash linked chain. The result is a distributed ledger that is not controlled or managed by a
central entity powered by a network of connected computers that use a consensus
mechanism to reach agreement over shared data\cite{whatisblockchain}.

The first blockchain was the Bitcoin blockchain\cite{bitcoin}. Bitcoin uses a
proof-of-work consensus algorithm based on Adam Back's hashcash cost function.
Network nodes, commonly known as miners, solve partial hash collisions to propose blocks of transactions. Additionally, Bitcoin also introduced an economic mechanism as a
complement to cryptographic primitives to secure distributed systems by
rewarding miners with an economically valuble token - the
bitcoin cryptocurrency. Miners are economically motivated to help secure and add
new transactions to the blockchain.

\subsection{Ethereum}

Vitalik Buterin developed Ethereum as a solution to leverage the distributed
consensus capabilities of blockchains to create decentralized
applications. Ethereum is a blockchain with built-in Turing
complete programming language that allow users to write so called \textit{smart
  contracts} that define arbitrary state transition functions\cite{ethereum}.
Nodes in the Ethereum network run the Ethereum Virtual Machine (EVM). The value
proposition of smart contracts is the ability to define arbitrary rules and
agreements in a self-enforcing and self-executing program. As a result,
participants in a protocol or network can trust the automatic enforced execution
of code in the smart contract rather than trust some centralized entity.

Although security focused members of the computer science community have
expressed a fair amount of concern about the viability of Ethereum as a
blockchain used for smart contracts due to the large attack surface presented by
its Turing complete programming language, the stark reality is that the Ethereum
developer ecosystem is the most active of any other blockchain ecosystem.
Furthermore, various members of the community are actively researching methods
to better secure the Ethereum network including formal verification,
proof-of-stake as an alternative consensus algorithm to proof-of-work and smart
contract programming languages with stronger security guarantees.
Consequently, with these points in mind we decided to build the OpenCollab protocol on Ethereum.

% Local Variables:
% org-ref-default-bibliography: ../bib/background.bib
% End:
