\section{Related Work}

Although cryptocurrencies and blockchains have only recently captured the attention of academics and
business people, the technical foundations of these systems actually have a longer
history. Cryptocurrencies and blockchains use ideas from past work by computer science
researchers in the areas of distributed consensus, electronic money
and digital time-stamping. Additionally, as types of distributed systems, the
deployment of cryptocurrencies and blockchains in a public
setting and on a large scale provide interesting insights particularly when viewed in the
context of past distributed consensus research.

\subsection{Distributed Consensus}

The reliability of a distributed system depends on system proccesses to
reach consensus on particular values. A distributed system can be viewed as a
replicated state machine consisting of a state machine replicated across
multiple processes. The replication of state across many processes creates a level of redundancy that
can protect a system against failure due to a single faulty individual component. The
replicated state machine uses a deterministic state transition function to map a set
of inputs and the current state to a new state\cite{faulttolerant}. State
transitions are atomic such that they either occur completely or do not occur at
all, consistent such that they must be valid mappings of inputs and the current
state to a new state and durable such that once they occur, state is permanently
updated. Thus, we can consider state transition function inputs as transactions\cite{transactions}.

Although a distributed system can protect against failure due to a single faulty
individual component, it remains vulnerable to multiple process faults that
prevent the system from achieving consensus on a particular value due to
incorrect process behavior. Process faults come in a different varieties, but
two common types are fail-stop faults, where processes crash and other proccesses can detect the failure, and
Byzantine faults, where processes exhibit arbitrary and potentially
malicious behavior\cite{faulttolerant}. In comparison to solutions for fail-stop
faults, solutions for Byzantine faults require additional complexity because
faulty Byzantine processes can transmit conflicting information to other processes that might
not be immediately detected. Lamport et al. initially presented Byzantine fault
tolerance using the Byzantine Generals Problem, in which a number of Byzantine
generals attempt to coordinate an attack on an enemy city in the presence of
potentially traitorous generals. The traitorous generals can be characterized as
Byzantine processes in a computing context.\cite{byzantine}.

Lamport et al. offered a few solutions to the Byzantine Generals Problem,
one involving oral messages requiring fewer than $f$ traitorous generals given
$3f + 1$ generals, and another involving messages with unforgeably signatures
that allows an arbitrary number of traitorous generals. However, due to an
assumption that the absence of messages can always be detected, these solutions
are only applicable to synchronous environments. In synchronous environments,
system designers can make assumptions about the maximum delay of network
messages, but in asynchronous environments, system designers cannot make any
assumptions about network delays\cite{tendermint}. In asynchronous environments,
an adversary might have the power to schedule network messages. Consequently, the solutions
of Lamport et al. would not be sufficient in an asynchronous environment
because there is no guarantee of the detecting the absence of messages.

The FLP impossibility proof states that distributed consensus in an asynchronous
environment with deterministic processes is impossible if a single process can
crash\cite{FLP}. However, the proof does not preclude distributed consensus in
asynchronous environments with certain weak synchronous assumptions or with
randomization. One example of a Byzantine fault tolerant consensus
algorithm that uses weak synchronous assumptions by tweaking a timeout parameter is Practical Byzantine Fault Tolerance (PBFT) which offers system
resilience as long as there are fewer than $f$ Byzantine processes given $3f +
1$ total processes at any given point in time\cite{pbft}. A number of computer
science researchers have also explored Byzantine fault tolerant consensus
algorithms using randomization. Ben-Or offered one such algorithm that involves
processes using a register that is probabilistically set to either $0$ or $1$ to
decide on a binary value. The algorithm works with probability eventually
reaching $1$, but also requires fewer than $f$ Byzantine processes given $5f +1$
total processes\cite{freechoice}. We will observe in Section \ref{sec:blockchains}
that certain cryptocurrency systems and blockchains use a combination of weak
synchronous assumptions and randomization to achieve Byzantine fault tolerant
distributed consensus in asynchronous environments.

\subsection{Electronic Money}

The advent of the Internet and the mainstream adoption of computing devices
encouraged the development of many forms of electronic money by recording
balances electronically on devices.

In 1983, David Chaum introduced a cryptographic protocol for anonymous payments using
blind signatures. In Chaum's protocol, a bank uses its private key to sign
blinded tokens and payees accept signed tokens by clearing a
signed token with the bank\cite{blindsignatures}. The authenticity of tokens can
be guaranteed by verifying the bank's signature on tokens with the bank's public
key. The bank also does not know the identity of a payer when clearing a signed
token sent by a payee because the token was blinded to obfuscate the amount and sender the bank
originally signed the token. Chaum applied this protocol in his DigiCash project.

One of the flaws of DigiCash is that it can only offer durable
transactions in exchange for decreased anonymity. Users must present tokens to
the bank for verification or else they are vulnerable to double spending attacks
since electronic messages can easily be duplicated\cite{camp1995}. Furthermore,
reliance on the bank creates a bottleneck for system throughput and a central
point of failure. If a bank's private key is compromised, an attacker can use
the bank's private key to create counterfeit tokens.

Easily duplicated electronic messages leave electronic money systems vulnerable
to denial of service attacks. As a solution, Adam Back proposed using hashcash, a
easily verifiable, but difficult to compute cost function to mint
tokens\cite{hashcash}. The cost function or \textit{proof-of-work} is based on finding partial hash
collisions on the $k$-bit string $0^{k}$, for which the fastest known algorithm
is brute force meaning users must perform a certain amount of work in terms of
computing cycles to mint tokens. A proof-of-work requirement discourages
electronic message duplication by making message creation costly.

Nick Szabo highlighted the utility of proof-of-work for electronic money in his bit gold protocol. The protocol
uses a proof of work function to compute a string of bits that is timestamped in a distributed property title registry\cite{bitgold}. Users can
verify owernship of a string of bits in the title registry. In contrast with
DigiCash, bit gold allows valuable bits to be created,
transferred and stored without depending on a trusted third party. However, a
system implementing such a protocol was never implemented in practice.

\subsection{Digital Time-Stamping}

Widespread digitization of all types of documents brought many benefits to
society, but also introduced the question of how to prove the existence and time of creation or change of a
digital document.

In 1991, Haber and Stornetta presented a time-stamping method for digital
documents that consisted of certificates cryptographically signed by a time-stamping service. The
certificates contain the hash of the document as well as linking information from a
previous certificate which includes a hash of the previous certificate's linking
information\cite{haber1991}. The result is a hash linked chain of certificates
that prevents the faking of time-stamps.

Bayer, Haber and Stornetta extended this time-stamping method using merkle
trees. In the original time-stamping method, verification of a document
timestamp can require at most $N$ steps by following the chain links to a
time-stamp certificate that is trustworthy\cite{bayer1993}. Instead of linking $N$
hashes of documents, the hash values can be stored in a merkle tree.
Participants can record the hashes of their own documents and the sibling hash
values along the path from the document hash to the root of the merkle tree.
Consequently, verification can be done in at most $\lg N$ steps by presenting the
document hash and the $\lg N$ hashes on the path to the root. This modified
time-stamping approach reduces storage requirements and verification time.

\subsection{Blockchains}
\label{sec:blockchains}

Blockchains combine learnings from past work in distributed consensus,
electronic money and digital time-stamping. A blockchain is a type of distributed system with two key defining
characteristics. The first characteristic is that transactions are grouped into blocks. A
common optimization reminiscient of Bayer, Haber and
Stronetta's digital time-stamping method using merkle trees is to store the root of a merkle tree that contains
transactions in the block header. This optimization allows clients to easily
verify transactions solely using the merkle roots of transactions in downloaded block headers without storing all
the actual transactions. The second characteristic is that blocks
are linked by cryptographic hashes. As demonstrated by Haber and Stornetta's
work with hash linked digital timestamps, a hash linked chain of blocks prevents
tampering of blocks unless an adversary has majority control of the system such
that it can rewrite the entire hash linked chain. The result is a distributed ledger that is not controlled or managed by a
central entity powered by a network of connected computers that use a consensus
mechanism to reach agreement over shared data\cite{whatisblockchain}.

\subsection{Bitcoin}

The first blockchain was the Bitcoin blockchain\cite{bitcoin} powering the
Bitcoin cryptocurrency system. Unlike traditional distributed systems, Bitcoin
is deployed in a public setting without any participation permissions and
publicly known process identities. These conditions leave distributed systems
vulnerable to Sybil attacks consisting of
an adversary using multiple identities to influence consensus
decisions\cite{sybil}. Additionally, as a global cryptocurrency system, Bitcoin
operates over the public internet, an asynchronous environment. Consequently,

Bitcoin achieves Byzantine fault tolerant distributed consensus in an public, adversarial
and asynchronous environment using randomization in the form of a proof-of-work
consensus algorithm. Proof-of-work is based on solving moderately hard
cryptographic puzzles in order to prevent computationally bounded adversaries
from claiming many identites in the system\cite{anonymousByzantine}. More specifically, Bitcoin's
proof-of-work algorithm is based on Adam Back's hashcash cost function. A subset
of Bitcoin nodes, the processes of the larger distributed system, solve partial
hash collisions to append blocks of transactions to the Bitcoin blockchain and
update the state of the system. These nodes are commonly known as miners. This
type of consensus algorithm has become commonly known as eventually consistent
Nakamoto consensus\cite{nakamotoConsensus}. Bitcoin can achieve consensus as
long as an adversary does not control more than half of the total computing
power. Phrased in terms of Byzantine fault tolerance, Bitcoin can achieve
consensus as long as less than $f$ computing power is Byzantine of $2f + 1$
total computing power.

Bitcoin also complements this consensus algorithm with economic rewards to
incentivize miners to secure and maintain the Bitcoin blockchain. The first
transaction in a new block mints new economically valuable tokens, the Bitcoin
cryptocurrency, that is rewarded to the miner that successfully solved a partial
hash collision and added the new block\cite{bitcoin}. As a result, miners are
economically motivated behave honestly. It is important to also note that a number of
flaws in Bitcoin have been discovered over the years, but a discussion of these
flaws is outside the scope of this thesis and will be left for outside research.

\subsection{Ethereum}

Vitalik Buterin developed Ethereum as a solution to leverage the distributed
consensus capabilities of blockchains to create decentralized
applications. Ethereum is a blockchain with built-in Turing
complete programming language that allow users to write so called \textit{smart
  contracts} that define arbitrary state transition functions\cite{ethereum}.
Nodes in the Ethereum network run the Ethereum Virtual Machine (EVM). The value
proposition of smart contracts is the ability to define arbitrary rules and
agreements in a self-enforcing and self-executing program. As a result,
participants in a protocol or network can trust the automatic enforced execution
of code in the smart contract rather than trust some centralized entity.

Similar to Bitcoin, Ethereum uses a proof-of-work consensus algorithm. However,
the core Ethereum developers have announced plans to move toward a
proof-of-stake consensus algorithm in the future that would instead rely on
cryptocurrency holders as validators to append new blocks to the Ethereum blockchain,
incentivizing good behavior by rewarding honest validators and discouraging bad
behavior by penalizing dishonest validators\cite{casper}.

The role of Ethereum in the contributions for this thesis is discussed in more
detail in Section \ref{sec:background}.

% Local Variables:
% org-ref-default-bibliography: ../bib/related.bib
% End:
